\documentclass{article}
\usepackage{graphicx} % Required for inserting images
\usepackage{booktabs}
\usepackage{amsmath}
 \usepackage{multirow}
 \usepackage{amssymb}
\usepackage[a4paper, top=1in, bottom=1in, left=1in, right=1in]{geometry}
\usepackage{tikz}
\usepackage{hyperref}
\usepackage{dsfont}
\usepackage{pifont}% http://ctan.org/pkg/pifont\
\usepackage{caption}
\usepackage{natbib}
\usepackage[normalem]{ulem}
\newcommand{\stkout}[1]{\ifmmode\text{\sout{\ensuremath{#1}}}\else\sout{#1}\fi}

\newcommand{\xmark}{\ding{55}}%

\setlength{\parskip}{1em} % Adjusts spacing between paragraphs
\usetikzlibrary{arrows.meta, positioning}

\usepackage[most]{tcolorbox} % Load tcolorbox package
\tcbuselibrary{listingsutf8}

\newtcolorbox[auto counter, number within=section]{observation}[2][]{colframe=blue!80!black, colback=blue!10, coltitle=white, title={#2}, #1}

\newcommand{\intuitionA}[1]{%
  \begin{tcolorbox}[breakable, colback=blue!15!white, colframe=blue!25!black]
#1
  \end{tcolorbox}
}

\title{\textbf{Siblings paper} \\ Exploded logit}
\author{Javiera Gazmuri}
\date{\today}

\begin{document}

\maketitle


\section{Model}

\subsection{Utility}

\begin{equation}
    u_{ijk} = \omega_{y} u_{y_ij} + (1-\omega_{y})u_{o_ik} + \gamma \mathds{1}[j = k] + \varepsilon_{ijk}
\end{equation}
where $j$ is the school-assignment of the younger children $y$ in family $i$, and $k$ is the school-assignment of the older children $o$. $\omega_{y}$ is the weight that the family places on the utility of the younger children. 

\begin{equation}
    u_{y_ij} = \beta_{y} dist_{{y_i}j} + \delta_y quality_j  \stkout{ +  \varepsilon_{y_ij}}
\end{equation}

\begin{equation}
    u_{o_ik} = \beta_{o} dist_{{o_i}k} + \delta_o quality_k  \stkout{+ \varepsilon_{o_ik}}
\end{equation}

where $\beta_y$, $\beta_o$, $\delta_y$ and $\delta_o$ are the parameters that govern preferences of the younger and older siblings respectively, and $\varepsilon_{y}$ and $\varepsilon_{o}$ are idiosyncratic i.i.d. preference shocks.

\textcolor{red}{I'm ignoring the distance between schools for now.}

Assumptions:
\begin{itemize}
    \item When families report marginal applications, it is like they were only applying to one child.
    \item $\varepsilon_{ijk}$ are i.i.d. type-I extreme value.
\end{itemize}

\subsection{Probabilities}

Families with more than one common-school applied were asked between,
\begin{enumerate}
    \item[a)] Worst-joint (WJ)
    \item [b)] Best-older-solo (BOS) and best-younger-solo (BYS)
\end{enumerate}
Families with only one common-school applied were asked between,
\begin{enumerate}
    \item [a)] ``Best''-joint (BJ)
    \item [b)] BOS \& BYS
\end{enumerate}

Let's define, 
\begin{align*}
    V_{ijk} \equiv \omega_{y} u_{y_ij} + (1-\omega_{y})u_{o_ik} + \gamma \mathds{1}[j = k]
\end{align*}
Hence,
\begin{align*}
    V_i^J \equiv V_{ijj} \tag{Assigned together} \\
    V_i^S \equiv V_{ilk} \tag{Split assignment}
\end{align*}
The probability that family $i$ prefers being assigned together is,
\begin{align*}
    P_i \equiv \Pr \left[ V_i^J + \varepsilon_{ijj} > V_i^S + \varepsilon_{ilk}  \right] = \frac{\exp\left(V_i^J\right)}{\exp\left(V_i^J\right) + \exp\left(V_i^S\right)}
\end{align*}

\subsection{Log-likelihood: Joint vs Split}
Let \(y_i \in \{0,1 \} \) be the observed answer (1 is prefer joint, 0 is prefer split),
\[
L_i^{JS} (\theta) = P_i^{y_i} (1-P_i)^{1-y_i} \quad, \theta = (\beta_y,\beta_o,\delta_y,\delta_o,\omega_y, \gamma)
\]
Assuming independent families \(i = 1,...,N\)
\begin{align*}
    \mathcal{L}^{JS}(\theta) &= \prod^N_{i=1}L_i^{JS}(\theta) \\
    \ell^{JS}(\theta) &= \sum_{i=1}^N y_i \log P_i + (1-y_i) \log (1-P_i)
\end{align*}

\subsection{Marginal Likelihoods}

We extend the model to include marginal ranking information for each sibling. Let $\mathcal{J}_i^y$ and $\mathcal{J}_i^o$ denote the choice sets (schools applied to) for the younger and older siblings in family $i$, respectively.

\subsubsection{Younger Sibling Marginal}

For the younger sibling's marginal ranking, let $j^*_i \in \mathcal{J}_i^y$ denote the first-choice school (orden = 1). The probability of this choice follows the exploded logit form:

\[
P_i^y(j^*_i) = \frac{\exp(u_{y_i j^*_i})}{\sum_{j \in \mathcal{J}_i^y} \exp(u_{y_i j})}
\]

where $u_{y_i j} = \beta_y dist_{y_i j} + \delta_y quality_j$ is the utility of school $j$ for the younger sibling.

The marginal log-likelihood contribution from the younger sibling in family $i$ is:
\[
\ell_i^y(\theta) = \log P_i^y(j^*_i)
\]

\subsubsection{Older Sibling Marginal}

Similarly, for the older sibling with first-choice school $k^*_i \in \mathcal{J}_i^o$:

\[
P_i^o(k^*_i) = \frac{\exp(u_{o_i k^*_i})}{\sum_{k \in \mathcal{J}_i^o} \exp(u_{o_i k})}
\]

where $u_{o_i k} = \beta_o dist_{o_i k} + \delta_o quality_k$.

The marginal log-likelihood contribution is:
\[
\ell_i^o(\theta) = \log P_i^o(k^*_i)
\]

\subsection{Full Log-likelihood}

Assuming independence of the marginal choices conditional on type (with Type-I Extreme Value idiosyncratic shocks), the full likelihood for family $i$ is:

\[
L_i(\theta) = L_i^{JS}(\theta) \times L_i^{y}(\theta) \times L_i^{o}(\theta)
\]

In log-likelihood form across all $N$ families:
\begin{align*}
    \ell(\theta) &= \sum_{i=1}^N \left[ \ell_i^{JS}(\theta) + \ell_i^y(\theta) + \ell_i^o(\theta) \right] \\
    &= \ell^{JS}(\theta) + \sum_{i=1}^N \ell_i^y(\theta) + \sum_{i=1}^N \ell_i^o(\theta)
\end{align*}

\textbf{Implementation note:} The extended model uses three data sources:
\begin{itemize}
    \item \texttt{survey\_responses.csv}: Joint vs split scenarios (BJ, WJ, BOS, BYS)
    \item \texttt{marginal\_applications\_older.csv}: Complete rankings for older sibling
    \item \texttt{marginal\_applications\_younger.csv}: Complete rankings for younger sibling
\end{itemize}

\section{Data}

From survey responses, we have 4 types of schools:
\begin{enumerate}
    \item Best-joint (BJ)
    \item Worst-joint (WJ)
    \item Best-older-solo (BOS)
    \item Best-younger-solo (BYS)
\end{enumerate}
From application data, we can say:
\begin{itemize}
    \item survey showed 10 options max for each, so some lists were trimmed (0,5\% sample approx).
    \item BOS and BYS: 6\% selected schools were not found in application data.
\end{itemize}
Characteristics of schools:
\begin{itemize}
    \item Specific for each sibling, even for BJ/WJ (since lat-long not always the same between siblings and between campus, quality depends on grade).
    \item 3\% - 6\% schools without info of school quality.
    \item 6\% we don't have information of distance for BOS/BYS
\end{itemize}

\section{Results}

$N= 9,231$ (from original $12,917$). 



\end{document}